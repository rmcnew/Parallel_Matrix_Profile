\documentclass[conference]{IEEEtran}
\usepackage{graphicx}

\begin{document}

\title{Project Midpoint Report:\\Matrix Profile MPI Implementation}

\author{\IEEEauthorblockN{Richard McNew\IEEEauthorrefmark{1}, Brody Larsen\IEEEauthorrefmark{2}, Feyisa Berisa\IEEEauthorrefmark{3} and Hugh Harps\IEEEauthorrefmark{4}}
\IEEEauthorblockA{Department of Computer Science, Utah State University\\ Logan, Utah, USA\\
\IEEEauthorrefmark{1}a02077329@usu.edu, \IEEEauthorrefmark{2}a01977457@usu.edu, \IEEEauthorrefmark{3}a01676072@usu.edu, \IEEEauthorrefmark{4}a02222128@usu.edu}
}


\maketitle
\begin{abstract}
MPI Implementation of the Matrix Profile
\end{abstract}

\begin{IEEEkeywords}
parallel computation, MPI, data science, Matrix Profile
\end{IEEEkeywords}

\section{Introduction}
Time series data is a collection of observations made sequentially in time.  Time series data is ubiquitous in our modern world and takes the form of sensor data, stock market data, network metrics, application logs, and many other forms.  Analyzing time series data is essential to understanding what is happening and enables informed decision-making and strategic planning.  One recent advancement in time series data analysis is the Matrix Profile \cite{MatrixProfile1}. 

The Matrix Profile is an amazing data structure and set of accompanying algorithms that annotate a time series and make most time series data mining problems easy to solve \cite{MatrixProfile2}. The Matrix Profile is:  1) exact - it allows for time series analysis without false positives or false negatives, 2) parameter-free - unlike many time series data analysis tools, no hyperparameter tuning is needed, 3) space efficient - a matrix profile data structure does not require much space, enabling large datasets to be processed in memory, 4) parallelizable - it is fast to compute on modern hardware, and 5) simple - it is easy to use and fairly easy to understand \cite{Keogh}.   

Most data science work is done in Python.  As a result, most Matrix Profile implementations in use today are written in Python \cite{Stumpy} and rely on the NumPy, SciPy, and Numba Python libraries for vector and matrix data types, numerical and scientific algorithms, and fast just-in-time optimizations.  Creating a Matrix Profile implementation using MPI in C++ will offer organizations that use MPI a way to use the Matrix Profile. 

\section{Project Thesis}
This project will create an MPI implementation of the Matrix Profile in C++.

\section{Background}

\section{Approach}

\section{Simulation}

\section{Tests}
In order to validate the proposed MPI implementation of Matrix Profile in C++ a test suite would need to be created.  The test suite will be created by finding or creating input time series data, running one or more Python Matrix Profile implementations against the input time series data, and capturing the output Matrix Profile data structures.  The MPI Matrix Profile implementation will be validated as correct if the output Matrix Profile data structures match those created by the Python Matrix Profile implementations. 

\section{Preliminary Results}
The MPI-based Matrix Profile implementation will produce the same results as Python-based Matrix Profile implementations and will work in MPI environments.


% ** Should the task lists and burndown charts be grouped like this or should
% ** we place the task list and burndown chart together by sprint, that is,
% ** Sprint 1 task list and burndown chart, Sprint 2 task list and burndown chart, etc.  ?

\section{Sprint Tasks}
%\begin{figure}
%\begin{center}
%\includegraphics[scale=0.75]{sprint_1_tasks.png}
%\caption{Sprint 1 Tasks}
%\end{center}
%\end{figure}

%\begin{figure}
%\begin{center}
%\includegraphics[scale=0.75]{sprint_2_tasks.png}
%\caption{Sprint 2 Tasks}
%\end{center}

%\end{figure}
%\begin{figure}
%\begin{center}
%\includegraphics[scale=0.75]{sprint_3_tasks.png}
%\caption{Sprint 3 Tasks}
%\end{center}

%\end{figure}
%\begin{figure}
%\begin{center}
%\includegraphics[scale=0.75]{sprint_4_tasks.png}
%\caption{Sprint 4 Tasks}
%\end{center}

%\end{figure}
%\begin{figure}
%\begin{center}
%\includegraphics[scale=0.75]{sprint_5_tasks.png}
%\caption{Sprint 5 Tasks}
%\end{center}
%\end{figure}


\section{Midpoint Report Burndown Charts}
Burndown charts go here.

%\begin{figure}
%\begin{center}
%\includegraphics[scale=0.75]{sprint_1_burndown_chart.png}
%\caption{Sprint 1 Burndown Chart}
%\end{center}
%\end{figure}

%\begin{figure}
%\begin{center}
%\includegraphics[scale=0.75]{sprint_2_burndown_chart.png}
%\caption{Sprint 2 Burndown Chart}
%\end{center}
%\end{figure}

%\begin{figure}
%\begin{center}
%\includegraphics[scale=0.75]{sprint_3_burndown_chart.png}
%\caption{Sprint 3 Burndown Chart}
%\end{center}
%\end{figure}

%\begin{figure}
%\begin{center}
%\includegraphics[scale=0.75]{sprint_4_burndown_chart.png}
%\caption{Sprint 4 Burndown Chart}
%\end{center}
%\end{figure}

%\begin{figure}
%\begin{center}
%\includegraphics[scale=0.75]{sprint_5_burndown_chart.png}
%\caption{Sprint 5 Burndown Chart}
%\end{center}
%\end{figure}


\bibliographystyle{IEEEtran}

% number is the number of reference labels
\begin{thebibliography}{7}  

\bibitem{MatrixProfile1} Yeh, Chin-Chia Michael, et al. (2016) "Matrix Profile I: All Pairs Similarity Joins for Time Series: A Unifying View that Includes Motifs, Discords, and Shapelets". ICDM:1317-1322.

\bibitem{MatrixProfile2} Zhu, Yan, et al. (2016) "Matrix Profile II: Exploiting a Novel Algorithm and GPUs to Break the One Hundred Million Barrier for Time Series Motifs and Joins". ICDM:739-748.

\bibitem{MatrixProfile14} Zimmerman, Zachary, et al. "Matrix Profile XIV: Scaling Time Series Motif Discovery with GPUs to Break a Quintillion Pairwise Comparisons a Day and Beyond." Proceedings of the ACM Symposium on Cloud Computing. 2019.

\bibitem{DynamicTimeWarping} T. Rakthanmanon et al., “Searching and Mining Trillions of Time Series Subsequences under Dynamic Time Warping,” In KDD 2012, 262-270.

\bibitem{Keogh} E. Keogh, \emph{The UCR Matrix Profile Page}, 2021. [Online]. Available: https://www.cs.ucr.edu/~eamonn/MatrixProfile.html.

\bibitem{Stumpy} S.M. Law, (2019). STUMPY: A Powerful and Scalable Python Library for Time Series Data Mining. Journal of Open Source Software, 4(39), 1504.

\bibitem{Pacheco} P. Pacheco. \emph{Parallel Programming with MPI}, Morgan Kaufmann; 1st edition (October 15, 1996), ISBN: 978-1558603394.

\end{thebibliography}

\end{document}
